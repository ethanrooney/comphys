\documentclass{article}

\usepackage{graphicx}
\usepackage{amsmath}
\usepackage{siunitx}
\usepackage{placeins}

\usepackage[margin=1in]{geometry}
\usepackage{float}


\def\hwtitle{Computational Physics HW6}
\def\hwauthor{Ethan Rooney}
\def\hwdate{2020-03-25}

\usepackage{fancyhdr}
\lhead{\hwauthor}
\chead{\hwtitle}
\rhead{\hwdate}
\lfoot{\hwauthor}
\cfoot{}
\rfoot{\thepage}
\renewcommand{\footrulewidth}{0.4pt}
\pagestyle{fancy}

\author{\hwauthor}
\title{\hwtitle}
\date{\hwdate}

\begin{document}

\maketitle
\thispagestyle{fancy}

\section{Introduction}
 
Using Linear Algebra to develop Polynomials of best fit for given experimental data points.

This week we use some standard statistical tools to fit models to data and to quantize how well our model fits the data.

\section{Results}

\subsection{Question 1}

With the first 6 seconds, or first 13 data points, of \texttt{cannonball.dat} input into \texttt{hw6} we get \texttt{16380.416133	4.573787	-5.526405	6.906034} when put into the form:

\[
	y = y_0 + v_0 t + \frac{g}{2}t^2
.\] 

We find:
\[ 
g=-11.05281 \frac{\text{m}}{\text{s}^2}
.\]

With an \[
\chi^2 = 6.906034
.\] 


This gives us a confidence interval of 0.734285

This is a reasonable fit to the data.

\subsection{Question 2}

For the entire data set of \texttt{cannonball.dat} we find \texttt{16395.644750	-3.424541	-4.692395	140.237263}. 

We find:
\[ 
g=-9.38479 \frac{\text{m}}{\text{s}^2}
.\]

With an \[
\chi^2 = 140.237263
.\] 


This gives us a confidence interval of 0.0548956

140.23 is at extreme end of a $\chi^2$ distribution, with only 5\% of fits more extreme, but leaves us inside of the magic number of 95\% confidence interval to be able to accept the underlying model.

\subsection{Question 3 Bonus}
For the Bonus problem we are trying to fit the following model:
\[
	A = A_00.5^\frac{t}{\tau}
.\] 

To do this we can linearise the model by taking the $\ln()$ of both sides.

\[
	\ln(A) = \ln(A_0) + t\frac{1}{\tau}\ln(0.5)
.\] 

As for the sigma, when working with radiation, we get a Poisson distribution, and the sigma of a Poisson distribution in a high count regime is $ \sqrt{N} $ and after the linearisation of the model by taking the log,$ \sigma = \frac{\sqrt{N}}{N}  $

After adjusting the program to handle the two above accommodations we get the following output \texttt{11.802353	-0.086268	12.190726}.

Where 
\[11.802353 = \ln(A_0).\]

\[-0.086268=\frac{1}{\tau}\ln(0.5)\]

\[12.190726 = \chi^2\]

Solving for the parameters we were trying to find we get $A_0 = 133519$. Which is very close to the known value of $133611$ and $\tau= 8.034 \text{days}$

The radioisotope responsible for the observed activity is most likely Iodine-131. A common daughter isotope of U-235 fission with a half-life of 8.02 days\footnote{http://hpschapters.org/northcarolina/NSDS/131IPDF.pdf}.

The biggest approximation that I made was the assumption that the $\sigma$ of a Poisson is $\approx \sqrt{N}$ in a high counts regime. For the later data points it is debatable if 200 counts is "high count" or not. 

This approximation is unlikely to affect my model significantly.

\section{Conclusion}

Well, we did it. The bonus problem was a fun brain teaser to think through.

\end{document}

